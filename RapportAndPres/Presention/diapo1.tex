\documentclass{beamer}
\mode<presentation> {
  \usetheme{Darmstadt}
}

\usepackage{graphicx} 
% Required package
\usepackage{animate}
\usepackage{booktabs}
\usepackage{tikz}
\usepackage{amsmath}
\usetikzlibrary{tikzmark}
\usepackage{multicol}
% Add necessary packages
\usepackage{graphicx}

% set captions with numbers
\setbeamertemplate{caption}[numbered]

% Page numbering setup
\setbeamertemplate{footline}{%
	\hspace*{\fill}%
	\usebeamercolor[fg]{page number in head/foot}%
	\usebeamerfont{page number in head/foot}%
	\insertframenumber\,/\,\inserttotalframenumber\kern1em\vskip2pt%
}


\title[2023-2024]{Simple Modeling of Road Traffic.} 

\author{Florent Gerbaud \\ Fatima Ezzahra Rharrour \\ MAM4} 
\institute[Polytech Nice-Sophia] % Your institution as it will appear on the bottom of every slide, may be shorthand to save space
{
Supervisor: \\ % Your institution for the title page
\medskip
{Didier Auroux} % Your email address
}
\date{\today} % Date, can be changed to a custom date
\titlegraphic{\includegraphics[width=3cm]{Polytech.png}} %
%-----------------------------------------------------------------------------------------
\AtBeginSection[]
{
	\begin{frame}
		\frametitle{Table of Contents}
		\tableofcontents[currentsection]
	\end{frame}
}
\begin{document}

\begin{frame}
\titlepage 
\end{frame}

\begin{frame}
\frametitle{Table of contents} 
\tableofcontents
\end{frame}

%----------------------------------------------------------------------------------------
%	PRESENTATION SLIDES
%----------------------------------------------------------------------------------------

%------------------------------------------------
\section{Introduction:} 
%------------------------------------------------
\begin{frame}{What is Road Traffic Modelling?}
	\begin{columns}
		\column{0.6\textwidth}
		\begin{itemize}
			\item Representing complex dynamics of vehicles moving along roads.
			\item Creating mathematical and computer models for:
			\begin{itemize}
				\item Understanding vehicle flow.
				\item Predicting movement patterns.
				\item Analyzing interactions on roads and highways.
			\end{itemize}
		\end{itemize}
		
		\column{0.4\textwidth}
		\begin{center}
			\centering
			\animategraphics[loop,width=3cm]{10}{Intro/test-}{0}{97}
		\end{center}
	\end{columns}
\end{frame}

%******************************************************************************************
\begin{frame}{The Objective of SMRT}
	\begin{block}{Key benefits of Road Traffic Modeling:}
		\begin{itemize}
	\item \textbf{Avoiding traffic jams:} Helps find solutions to prevent traffic jams on roads.
	\item \textbf{Making Roads Better:} Finds ways to improve roads and make them work smoother.
	\item \textbf{Understanding how traffic works:} Helps figure out how different things affect traffic and predict what might happen.
	\item \textbf{Making transportation better:} Shows how well transportation works and helps make it even better.
	\item \textbf{Saving time and money:} Aims to reduce time spent waiting in traffic and the money spent on each trip.
\end{itemize}
	\end{block}
\end{frame}
%******************************************************************************************

%******************************************************************************************
\begin{frame}{Project organization overview and Useful Definition}
	\begin{center}
    	\includegraphics[width=0.7\textwidth]{org1.png} 
    \end{center}
    \begin{block}{Ordinary Differential Equation (ODE) :}
    	An ODE is a mathematical equation that relates a function to its derivatives with respect to one or more independent variables.
    	\[
    	F\left(x,y,y',\ldots ,y^{(n-1)}\right)=y^{(n)}
    	\]
    \end{block}
\end{frame}
%******************************************************************************************



%------------------------------------------------
\section{Study of the discret case:}
%------------------------------------------------
\subsection{Ordinary Differential Equation}
\begin{frame}{Ordinary Differential Equation (Theory):}

\begin{figure}
    \centering
    \includegraphics[width=0.6\textwidth]{discret.png} 
    \caption{ Discret Model.}
\end{figure}
\vspace{-0.7cm}
\begin{block}{ODE to solve}
	\begin{center}
		$\boxed{\mathbf{y'(t) = f{\bigl (}t, y(t){\bigr )}}}$
	\end{center}
	Euler Explicit method to numerically solve the solutions:
	\begin{itemize}
		\item First step of the resolution: $\boxed{y_0 = y(t_0)}$.
		\item Recursive process to find the n-th solution of the ODE: $\boxed{y_{n+1} = y_{n} + hf(t_{n}, y_{n})}$
	\end{itemize}
\end{block}


    
\end{frame}


\begin{frame}{The Linear Approach}
	%\vspace{-0.5cm}
	\begin{figure}
		\includegraphics[width=1.0\textwidth]{test.png}
	\end{figure}
\end{frame}

\begin{frame}{The Newell Approach}
		\begin{figure}
			\includegraphics[width=1.0\textwidth]{test2.png}
		\end{figure}
\end{frame}
\subsection{Simulations}
\begin{frame}{Accordion phenomenon}
	%\subsection{Simulations}
	\begin{minipage}{0.48\textwidth}
		\centering
		\begin{figure}
			\includegraphics[width=\textwidth]{Accordeon1.png}
			\caption{Modelisation of the accordion phenomenon with The Linear Model}
			\label{fig:AL}
		\end{figure}
	\end{minipage}\hfill
	\begin{minipage}{0.48\textwidth}
		\centering
		\begin{figure}
			\includegraphics[width=\textwidth]{1W2_Accord.png}
			\caption{Modelisation of the accordion phenomenon with The Newell's Model}
			\label{fig:AN}
		\end{figure}
	\end{minipage}
	\begin{block}{}
		We could see the difference of modelisation and realism between the Linear model (figure \ref{fig:AL}) and the Newell's model (figure \ref{fig:AN})
	\end{block}
\end{frame}
\begin{frame}{Drunk drivers }
	\begin{minipage}{0.49\textwidth}
		\centering
		\begin{figure}
			\includegraphics[width=1.0\textwidth]{Model1W3C_O_Aco_D2_Linear.png}
			\caption{Simulation of Traffic Flow with one drunk driver (Linear Model)}
			\label{fig:DL}
		\end{figure}
	\end{minipage}\hfill
	\begin{minipage}{0.49\textwidth}
		\centering
		\begin{figure}
			\includegraphics[width=1.1\textwidth]{Model1W3C_O_Aco_D2_Newell.png}
			\caption{Simulation of Traffic Flow with one drunk driver (Newell's Model)}
			\label{fig:DN}
		\end{figure}
	\end{minipage}
	\begin{block}{}
		It is interesting to note that with Newell's Method (Figure \ref{fig:DN}), the variations are "smoothed and much less significant than in the case of the linear method (\ref{fig:DL})."
	\end{block}
\end{frame}

\begin{frame}{Accident phenomenon}
    \begin{figure}
    	\includegraphics[width=0.5\textwidth]{1W2_Acc2.png}
    	\caption{Modelisation of the accordion phenomenon with The Newell's Model}
    	\label{fig:ACC}
    \end{figure}
    \begin{block}{}
    	On the figure \ref{fig:ACC}, you can see that when the curves intersect, there is an accident
    \end{block}
\end{frame}
%Cas des 3 voitures !!
\subsection{Study of Equilibrium and Stability}
\begin{frame}{Analytical Solutions for the Linear Model}
	\begin{figure}[H]
		\centering
		\begin{minipage}[t]{0.49\linewidth}
			\centering
			\includegraphics[width=\linewidth]{Stability.png}
			\caption{Analytical Solution for two Cars}
			\label{fig:StabilityAnalysis}
		\end{minipage}
		\begin{minipage}[t]{0.49\linewidth}
			\centering
			\includegraphics[width=\linewidth]{AnalyticalSolution.png}
			\caption{Analytical Solution for three Cars}
			\label{fig:AnalyticalSolution}
		\end{minipage}\hfill
		\label{fig:CombinedFigures}
	\end{figure}
\end{frame}

\begin{frame}{Vector Field for Newell's model}
	\begin{figure}[H]
		\centering
		\begin{minipage}[t]{0.53\textwidth}
				\centering
				\includegraphics[width=\linewidth]{VectorFIeld.png}
				\caption{Stability for 2 cars}
		\end{minipage}
		\hfill 
		\begin{minipage}[t]{0.44\textwidth}
			\centering
			\includegraphics[width=0.9\linewidth]{FieldOfVector_CV2.png}
			\caption{Stability for 3 cars}
			\label{fig:FV1}
		\end{minipage}
		\label{fig:CombinedFigures}
	\end{figure}
\end{frame}



\section{Study of the continuous model:}
\begin{frame}{Model used in the Macroscopic Model}
	\begin{alertblock}{Conservation Law}
		\begin{itemize}
			\item $\partial_t\rho + \partial_x\left[ \rho\left( 1-\frac{\rho}{\rho_{\text{max}}}\right) \cdot V_{\text{max}}\right] = 0 $
		\end{itemize}
	\end{alertblock}
	\begin{block}{Initial and Boundary Conditions}
		\begin{itemize}
			\item $\rho(x,0) = \rho_0(x), \quad x \in \Omega,$
			\item $\rho(0,t) = \rho(L,t), \quad t \geq 0$
			\item $\Omega := \left] 0,L\right[, $
			\item $\rho(x,t)$ represents the traffic density at position $x$ and time $t$, 
		\end{itemize}
	\end{block}
\end{frame}

\subsection{Euler Explicit Method}
\begin{frame}{First Numerical Scheme to perform the Solution of the Equation}
	\begin{alertblock}{Numerical Scheme}
		\begin{itemize}
			\item $\rho_{i}^{n+1} = \rho_i^n - \frac{\Delta t}{\Delta x} \cdot \left(\rho_i^n \cdot v_i^n - \rho_{i-1}^n \cdot v_{i-1}^n \right)
					=0$
		
			\item $v_i^n = \left( 1 - \frac{\rho_i^n}{\rho_{max}}\right)  \times V_{max}$
		\end{itemize}
	\end{alertblock}
	\begin{block}{Initial Condition and Discretization}
		\[
		\boxed{
			\begin{aligned}
				&\rho_0(x)=0.2 \cdot \sin\left(2 \cdot \pi \cdot \frac{x}{L}\right) + 0.3 \\
				&\Delta t = 0.01 \\
				&\Delta x = 1
			\end{aligned}
		}
		\]
	\end{block}
	%slide : 
%Partie Math EDP
%Equation + cond au bord
%petite def de ce qu’il y a dans l’équation
\end{frame}

\begin{frame}{Results obtained with the Euler-Explicit Method}
	\vspace{-0.5cm}
	\begin{minipage}[t]{0.48\linewidth}
		\begin{figure}
			\centering
			\includegraphics[width=\linewidth]{traffic_flow_density_map.png}
			\caption[Traffic Flow Simulation With Euler Explicit]{ This figure illustrates the solution of the PDE at any time and position.}
			\label{fig:traffic_flow_density_map}
		\end{figure}
	\end{minipage}
	\hfill
	\begin{minipage}[t]{0.48\linewidth}
		\begin{figure}
			\centering
			\animategraphics[loop,width=\linewidth]{20}{AnimEE/EE-}{0}{1000}
			\caption{The Animation for the simulation with Euler-Explicit Method}
			\label{fig:euler_explicit_animation}
		\end{figure}
	\end{minipage}
	\vspace{-0.2cm}
	\begin{block}{}
		\begin{multicols}{2}
			\begin{itemize}
				\item Scheme Diffusivity
				\item Periodic Boundary Conditions 
				\item Recreation of new peaks of density
				\item Traffic moves forward
				% Add more items as needed
			\end{itemize}
		\end{multicols}
	\end{block}
\end{frame}


\subsection{Lax-Friedrichs method}
\begin{frame}{A new Scheme to Represent the High Density of the Traffic.}
	\begin{alertblock}{Numerical Scheme}
		\begin{equation*}
			\begin{split}
				\rho_{j}^t &= \frac{1}{2} \left(\rho_{j+1}^{t-1} + \rho_{j-1}^{t-1}\right) 
			 - \frac{\Delta t}{2 \cdot \Delta x} \left( \rho_{j+1}^{t-1} \left(1 - \frac{\rho_{j+1}^{t-1}}{\rho_{max}}\right) \cdot V_{\text{max}} \right. \\
				&\quad \left. - \rho_{j-1}^{t-1} \left(1 - \frac{\rho_{j-1}^{t-1}}{\rho_{max}}\right) \cdot V_{\text{max}} \right)
			\end{split}
		\end{equation*}
		
	\end{alertblock}
	\begin{block}{Initial Condition and Discretization}
		\[
		\boxed{
			\begin{aligned}
				&\rho_0(x)=0.2 \cdot \sin\left(2 \cdot \pi \cdot \frac{x}{L}\right) + 0.8 \\
				&\Delta t = 0.01 \\
				&\Delta x = 1
			\end{aligned}
		}
		\]
	\end{block}
	
	%slide: 
	%Partie Math EDP
	%Equation + cond au bord
	%petite définition de ce qu’il y a dans l’équation
\end{frame}


\begin{frame}{Results obtained with the Lax-Friedrichs Method}
	\vspace{-0.4cm}
	\begin{minipage}[t]{0.48\linewidth}
		\begin{figure}
			\centering
		\includegraphics[width=\linewidth]{traffic_flow_density_map_LF.png}
		\caption{ This figure illustrates the solution of the PDE at any time and position.}
		\label{fig:traffic_flow_density_map_LF}
		\end{figure}
	\end{minipage}
	\hfill
	\begin{minipage}[t]{0.48\linewidth}
		\begin{figure}
			\centering
			\animategraphics[loop,width=\linewidth]{20}{AnimLF/LF-}{0}{1000}
			\caption{The Animation for the simulation with Lax-Friedrichs Method}
			\label{fig:lax_friedrichs_animation}
		\end{figure}
	\end{minipage}
	\vspace{-0.2cm}
	\begin{block}{}
		\begin{multicols}{2}
			\begin{itemize}
				\small
				\item Scheme Diffusivity/Dispersivity
				\item Periodic Boundary Conditions 
				\item Recreation of new peaks of density
				\item Traffic moves backward
				% Add more items as needed
			\end{itemize}
		\end{multicols}
	\end{block}
\end{frame}

\section{Conclusion}

\end{document}